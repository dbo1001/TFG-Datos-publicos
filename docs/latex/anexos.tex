\documentclass[a4paper,12pt,twoside]{memoir}

% Castellano
\usepackage[spanish,es-tabla]{babel}
\selectlanguage{spanish}
\usepackage[utf8]{inputenc}
\usepackage[T1]{fontenc}
\usepackage{lmodern} % scalable font
\usepackage{microtype}
\usepackage{placeins}
\usepackage{listings}
\usepackage{eurosym}
\usepackage{longtable,booktabs}
\usepackage[export]{adjustbox}
\usepackage{dirtree}

\RequirePackage{booktabs}
\RequirePackage[table]{xcolor}
\RequirePackage{xtab}
\RequirePackage{multirow}

% Multi-page tables using
\usepackage{longtable}
\usepackage{tabularx}

% Muestra un directorio en el árbol
\newcommand{\desc}[1]{
	\ldots{} \begin{minipage}[t]{8cm}
		#1{.}
	\end{minipage}
}

% Cell with line break (e.g. \specialcell{Foo\\bar})
\newcommand{\specialcell}[2][c]{%
	\begin{tabular}[#1]{@{}l@{}}#2\end{tabular}}

% Links
\usepackage[colorlinks]{hyperref}
\hypersetup{
	allcolors = {red}
}

% Ecuaciones
\usepackage{amsmath}

% Rutas de fichero / paquete
\newcommand{\ruta}[1]{{\sffamily #1}}

% Párrafos
\nonzeroparskip


% Imagenes
\usepackage{graphicx}
\newcommand{\imagen}[2]{
	\begin{figure}[!h]
		\centering
		\includegraphics[width=0.9\textwidth]{#1}
		\caption{#2}\label{fig:#1}
	\end{figure}
	\FloatBarrier
}
\newcommand{\imagencite}[3]{
	\begin{figure}[!h]
		\centering
		\includegraphics[width=0.9\textwidth]{#1}
		\caption[#2]{#2 \cite{#3}}\label{fig:#1}
	\end{figure}
	\FloatBarrier
}
\newcommand{\imagengrande}[2]{
	\begin{figure}[!h]
		\centering
		\includegraphics[width=1.2\textwidth,center]{#1}
		\caption{#2}\label{fig:#1}
	\end{figure}
\FloatBarrier
}

\newcommand{\imagenflotante}[2]{
	\begin{figure}%[!h]
		\centering
		\includegraphics[width=0.9\textwidth]{#1}
		\caption{#2}\label{fig:#1}
	\end{figure}
}

% Pie de página con enlace
\newcommand{\foothref}[2]{
	\href{#2}{#1}\footnote{#1: \href{#2}{#2}}
}



% El comando \figura nos permite insertar figuras comodamente, y utilizando
% siempre el mismo formato. Los parametros son:
% 1 -> Porcentaje del ancho de página que ocupará la figura (de 0 a 1)
% 2 --> Fichero de la imagen
% 3 --> Texto a pie de imagen
% 4 --> Etiqueta (label) para referencias
% 5 --> Opciones que queramos pasarle al \includegraphics
% 6 --> Opciones de posicionamiento a pasarle a \begin{figure}
\newcommand{\figuraConPosicion}[6]{%
  \setlength{\anchoFloat}{#1\textwidth}%
  \addtolength{\anchoFloat}{-4\fboxsep}%
  \setlength{\anchoFigura}{\anchoFloat}%
  \begin{figure}[#6]
    \begin{center}%
      \Ovalbox{%
        \begin{minipage}{\anchoFloat}%
          \begin{center}%
            \includegraphics[width=\anchoFigura,#5]{#2}%
            \caption{#3}%
            \label{#4}%
          \end{center}%
        \end{minipage}
      }%
    \end{center}%
  \end{figure}%
}

%
% Comando para incluir imágenes en formato apaisado (sin marco).
\newcommand{\figuraApaisadaSinMarco}[5]{%
  \begin{figure}%
    \begin{center}%
    \includegraphics[angle=90,height=#1\textheight,#5]{#2}%
    \caption{#3}%
    \label{#4}%
    \end{center}%
  \end{figure}%
}
% Para las tablas
\newcommand{\otoprule}{\midrule [\heavyrulewidth]}
%
% Nuevo comando para tablas pequeñas (menos de una página).
\newcommand{\tablaSmall}[5]{%
 \begin{table}
  \begin{center}
   \rowcolors {2}{gray!35}{}
   \begin{tabular}{#2}
    \toprule
    #4
    \otoprule
    #5
    \bottomrule
   \end{tabular}
   \caption{#1}
   \label{tabla:#3}
  \end{center}
 \end{table}
}

%
%Para el float H de tablaSmallSinColores
\usepackage{float}

%
% Nuevo comando para tablas pequeñas (menos de una página).
\newcommand{\tablaSmallSinColores}[5]{%
 \begin{table}[H]
  \begin{center}
   \begin{tabular}{#2}
    \toprule
    #4
    \otoprule
    #5
    \bottomrule
   \end{tabular}
   \caption{#1}
   \label{tabla:#3}
  \end{center}
 \end{table}
}

\newcommand{\tablaApaisadaSmall}[5]{%
\begin{landscape}
  \begin{table}
   \begin{center}
    \rowcolors {2}{gray!35}{}
    \begin{tabular}{#2}
     \toprule
     #4
     \otoprule
     #5
     \bottomrule
    \end{tabular}
    \caption{#1}
    \label{tabla:#3}
   \end{center}
  \end{table}
\end{landscape}
}

%
% Nuevo comando para tablas grandes con cabecera y filas alternas coloreadas en gris.
\newcommand{\tabla}[6]{%
  \begin{center}
    \tablefirsthead{
      \toprule
      #5
      \otoprule
    }
    \tablehead{
      \multicolumn{#3}{l}{\small\sl continúa desde la página anterior}\\
      \toprule
      #5
      \otoprule
    }
    \tabletail{
      \hline
      \multicolumn{#3}{r}{\small\sl continúa en la página siguiente}\\
    }
    \tablelasttail{
      \hline
    }
    \bottomcaption{#1}
    \rowcolors {2}{gray!35}{}
    \begin{xtabular}{#2}
      #6
      \bottomrule
    \end{xtabular}
    \label{tabla:#4}
  \end{center}
}

%
% Nuevo comando para tablas grandes con cabecera.
\newcommand{\tablaSinColores}[6]{%
  \begin{center}
    \tablefirsthead{
      \toprule
      #5
      \otoprule
    }
    \tablehead{
      \multicolumn{#3}{l}{\small\sl continúa desde la página anterior}\\
      \toprule
      #5
      \otoprule
    }
    \tabletail{
      \hline
      \multicolumn{#3}{r}{\small\sl continúa en la página siguiente}\\
    }
    \tablelasttail{
      \hline
    }
    \bottomcaption{#1}
    \begin{xtabular}{#2}
      #6
      \bottomrule
    \end{xtabular}
    \label{tabla:#4}
  \end{center}
}

%
% Nuevo comando para tablas grandes sin cabecera.
\newcommand{\tablaSinCabecera}[5]{%
  \begin{center}
    \tablefirsthead{
      \toprule
    }
    \tablehead{
      \multicolumn{#3}{l}{\small\sl continúa desde la página anterior}\\
      \hline
    }
    \tabletail{
      \hline
      \multicolumn{#3}{r}{\small\sl continúa en la página siguiente}\\
    }
    \tablelasttail{
      \hline
    }
    \bottomcaption{#1}
  \begin{xtabular}{#2}
    #5
   \bottomrule
  \end{xtabular}
  \label{tabla:#4}
  \end{center}
}



\definecolor{cgoLight}{HTML}{EEEEEE}
\definecolor{cgoExtralight}{HTML}{FFFFFF}

%
% Nuevo comando para tablas grandes sin cabecera.
\newcommand{\tablaSinCabeceraConBandas}[5]{%
  \begin{center}
    \tablefirsthead{
      \toprule
    }
    \tablehead{
      \multicolumn{#3}{l}{\small\sl continúa desde la página anterior}\\
      \hline
    }
    \tabletail{
      \hline
      \multicolumn{#3}{r}{\small\sl continúa en la página siguiente}\\
    }
    \tablelasttail{
      \hline
    }
    \bottomcaption{#1}
    \rowcolors[]{1}{cgoExtralight}{cgoLight}

  \begin{xtabular}{#2}
    #5
   \bottomrule
  \end{xtabular}
  \label{tabla:#4}
  \end{center}
}


% Plantilla para los casos de uso
\newcommand{\casoDeUso}[9]{
	\begin{longtable}[H]{@{}ll@{}}
		\toprule
		\begin{minipage}[b]{0.23\columnwidth}\raggedright\strut
			\textbf{CU-#1}\strut
		\end{minipage} & \begin{minipage}[b]{0.71\columnwidth}\raggedright\strut
			\textbf{#2}\strut
		\end{minipage}\tabularnewline
		\midrule
		\endhead
		\begin{minipage}[t]{0.23\columnwidth}\raggedright\strut
			\textbf{Versión}\strut
		\end{minipage} & \begin{minipage}[t]{0.71\columnwidth}\raggedright\strut
			1.0\strut
		\end{minipage}\tabularnewline
		\begin{minipage}[t]{0.23\columnwidth}\raggedright\strut
			\textbf{Autor}\strut
		\end{minipage} & \begin{minipage}[t]{0.71\columnwidth}\raggedright\strut
			\nombre\strut
		\end{minipage}\tabularnewline
		\begin{minipage}[t]{0.23\columnwidth}\raggedright\strut
			\textbf{Requisitos asociados}\strut
		\end{minipage} & \begin{minipage}[t]{0.71\columnwidth}\raggedright\strut
			#3\strut
		\end{minipage}\tabularnewline
		\begin{minipage}[t]{0.23\columnwidth}\raggedright\strut
			\textbf{Descripción}\strut
		\end{minipage} & \begin{minipage}[t]{0.71\columnwidth}\raggedright\strut
			#4\strut
		\end{minipage}\tabularnewline
		\begin{minipage}[t]{0.23\columnwidth}\raggedright\strut
			\textbf{Precondición}\strut
		\end{minipage} & \begin{minipage}[t]{0.71\columnwidth}\raggedright\strut
			#5\strut
		\end{minipage}\tabularnewline
		\begin{minipage}[t]{0.23\columnwidth}\raggedright\strut
			\textbf{Acciones}\strut
		\end{minipage} & \begin{minipage}[t]{0.71\columnwidth}\raggedright\strut
			\begin{enumerate}
				\def\labelenumi{\arabic{enumi}.}
				\tightlist
				#6
			\end{enumerate}\strut
		\end{minipage}\tabularnewline
		\begin{minipage}[t]{0.23\columnwidth}\raggedright\strut
			\textbf{Postcondición}\strut
		\end{minipage} & \begin{minipage}[t]{0.71\columnwidth}\raggedright\strut
			#7\strut
		\end{minipage}\tabularnewline
		\begin{minipage}[t]{0.23\columnwidth}\raggedright\strut
			\textbf{Excepciones}\strut
		\end{minipage} & \begin{minipage}[t]{0.71\columnwidth}\raggedright\strut
			\begin{itemize}
				\tightlist
				#8
			\end{itemize}\strut
		\end{minipage}\tabularnewline
		\begin{minipage}[t]{0.23\columnwidth}\raggedright\strut
			\textbf{Importancia}\strut
		\end{minipage} & \begin{minipage}[t]{0.71\columnwidth}\raggedright\strut
			#9\strut
		\end{minipage}\tabularnewline
		\bottomrule
		\caption{CU-#1 #2.}
		\label{CU:#1}
	\end{longtable}
	\newpage
}

\definecolor{mygreen}{rgb}{0,0.6,0}
\definecolor{mygray}{rgb}{0.5,0.5,0.5}
\definecolor{mymauve}{rgb}{0.58,0,0.82}


% Encoding lstlisting
\lstset{literate=
	{á}{{\'a}}1 {é}{{\'e}}1 {í}{{\'i}}1 {ó}{{\'o}}1 {ú}{{\'u}}1
	{Á}{{\'A}}1 {É}{{\'E}}1 {Í}{{\'I}}1 {Ó}{{\'O}}1 {Ú}{{\'U}}1
	{à}{{\`a}}1 {è}{{\`e}}1 {ì}{{\`i}}1 {ò}{{\`o}}1 {ù}{{\`u}}1
	{À}{{\`A}}1 {È}{{\'E}}1 {Ì}{{\`I}}1 {Ò}{{\`O}}1 {Ù}{{\`U}}1
	{ä}{{\"a}}1 {ë}{{\"e}}1 {ï}{{\"i}}1 {ö}{{\"o}}1 {ü}{{\"u}}1
	{Ä}{{\"A}}1 {Ë}{{\"E}}1 {Ï}{{\"I}}1 {Ö}{{\"O}}1 {Ü}{{\"U}}1
	{â}{{\^a}}1 {ê}{{\^e}}1 {î}{{\^i}}1 {ô}{{\^o}}1 {û}{{\^u}}1
	{Â}{{\^A}}1 {Ê}{{\^E}}1 {Î}{{\^I}}1 {Ô}{{\^O}}1 {Û}{{\^U}}1
	{œ}{{\oe}}1 {Œ}{{\OE}}1 {æ}{{\ae}}1 {Æ}{{\AE}}1 {ß}{{\ss}}1
	{ű}{{\H{u}}}1 {Ű}{{\H{U}}}1 {ő}{{\H{o}}}1 {Ő}{{\H{O}}}1
	{ç}{{\c c}}1 {Ç}{{\c C}}1 {ø}{{\o}}1 {å}{{\r a}}1 {Å}{{\r A}}1
	{€}{{\euro}}1 {£}{{\pounds}}1 {«}{{\guillemotleft}}1
	{»}{{\guillemotright}}1 {ñ}{{\~n}}1 {Ñ}{{\~N}}1 {¿}{{?`}}1
}

% Configuración de listing.
% De https://en.wikibooks.org/wiki/LaTeX/Source_Code_Listings
\lstset{ 
	backgroundcolor=\color{white},   % choose the background color; you must add \usepackage{color} or \usepackage{xcolor}; should come as last argument
	basicstyle=\footnotesize,        % the size of the fonts that are used for the code
	breakatwhitespace=false,         % sets if automatic breaks should only happen at whitespace
	breaklines=true,                 % sets automatic line breaking
	captionpos=b,                    % sets the caption-position to bottom
	commentstyle=\color{mygreen},    % comment style
	deletekeywords={...},            % if you want to delete keywords from the given language
	escapeinside={\%*}{*)},          % if you want to add LaTeX within your code
	extendedchars=true,              % lets you use non-ASCII characters; for 8-bits encodings only, does not work with UTF-8
	frame=single,	                 % adds a frame around the code
	keepspaces=true,                 % keeps spaces in text, useful for keeping indentation of code (possibly needs columns=flexible)
	keywordstyle=\color{blue},       % keyword style
	language=Python,                 % the language of the code
	morekeywords={*,...},            % if you want to add more keywords to the set
	numbers=none,                    % where to put the line-numbers; possible values are (none, left, right)
	numbersep=5pt,                   % how far the line-numbers are from the code
	numberstyle=\tiny\color{mygray}, % the style that is used for the line-numbers
	rulecolor=\color{black},         % if not set, the frame-color may be changed on line-breaks within not-black text (e.g. comments (green here))
	showspaces=false,                % show spaces everywhere adding particular underscores; it overrides 'showstringspaces'
	showstringspaces=false,          % underline spaces within strings only
	showtabs=false,                  % show tabs within strings adding particular underscores
	stepnumber=2,                    % the step between two line-numbers. If it's 1, each line will be numbered
	stringstyle=\color{mymauve},     % string literal style
	tabsize=2,	                     % sets default tabsize to 2 spaces
	title=\lstname                   % show the filename of files included with \lstinputlisting; also try caption instead of title
}

\graphicspath{ {./img/} }

% Capítulos
\chapterstyle{bianchi}
\newcommand{\capitulo}[2]{
	\setcounter{chapter}{#1}
	\setcounter{section}{0}
	\chapter*{#2}
	\addcontentsline{toc}{chapter}{#2}
	\markboth{#2}{#2}
}

% Apéndices
\renewcommand{\appendixname}{Apéndice}
\renewcommand*\cftappendixname{\appendixname}

\newcommand{\apendice}[1]{
	%\renewcommand{\thechapter}{A}
	\chapter{#1}
}

\renewcommand*\cftappendixname{\appendixname\ }

% Formato de portada
\makeatletter
\usepackage{xcolor}
\newcommand{\tutor}[1]{\def\@tutor{#1}}
\newcommand{\course}[1]{\def\@course{#1}}
\definecolor{cpardoBox}{HTML}{E6E6FF}
\def\maketitle{
  \null
  \thispagestyle{empty}
  % Cabecera ----------------
\noindent\includegraphics[width=\textwidth]{cabecera}\vspace{1cm}%
  \vfill
  % Título proyecto y escudo informática ----------------
  \colorbox{cpardoBox}{%
    \begin{minipage}{.8\textwidth}
      \vspace{.5cm}\Large
      \begin{center}
      \textbf{TFG del Grado en Ingeniería Informática}\vspace{.6cm}\\
      \textbf{\LARGE\@title{}}
      \end{center}
      \vspace{.2cm}
    \end{minipage}

  }%
  \hfill\begin{minipage}{.20\textwidth}
    \includegraphics[width=\textwidth]{escudoInfor}
  \end{minipage}
  \vfill
  % Datos de alumno, curso y tutores ------------------
  \begin{center}%
  {%
    \noindent\LARGE
    Presentado por \@author{}\\ 
    en Universidad de Burgos --- \@date{}\\
    Tutor: \@tutor{}\\
  }%
  \end{center}%
  \null
  \cleardoublepage
  }
\makeatother

% Cabecera no en mayúsculas
\nouppercaseheads

\newcommand{\nombre}{Iván Arjona Alonso} %%% cambio de comando

% Datos de portada
\title{Aplicación Web para la recopilación, tratamiento y visualización de datos públicos}
\author{\nombre}
\tutor{Dr. José Francisco Díez Pastor\\y Dr. Jesús Manuel Maudes Raedo}
\date{\today}

\begin{document}

\maketitle



\cleardoublepage



%%%%%%%%%%%%%%%%%%%%%%%%%%%%%%%%%%%%%%%%%%%%%%%%%%%%%%%%%%%%%%%%%%%%%%%%%%%%%%%%%%%%%%%%



\frontmatter


\clearpage

% Indices
\tableofcontents

\clearpage

\listoffigures

\clearpage

\listoftables

\clearpage

\mainmatter

\appendix

\apendice{Plan de Proyecto Software}

\section{Introducción}

\section{Planificación temporal}

El desarrollo del proyecto se ha llevado a cabo utilizando metodologias ágiles, basandose en la metodología \textit{scrum} con algunas modificaciones (una sóla persona y sin reuniones diarias).

Se aplicó una estrategia de desarrollo incremental, con iteraciones que llamaremos \textit{sprints}.

El resultado de cada iteración es un entregable, sobre el que se discute en la reunión posterior a cada sprint.

Se realizó, en principio, una reunión a la semana con los tutores para exponer las modificaciones realizadas en el sprint anterior y planificar los cambios a realizar en la siguiente iteración. 

Estas tareas están priorizadas por el tiempo estimado de su realización, se puede ver esta estimación en el enlace de cada sprint a sus tareas. 

A continuación se va a realizar un breve resumen de las tareas realizadas en cada una de las iteraciones, así como la duración de cada sprint y el gráfico \textit{burndown} correspondiente.

\subsection{Sprint 0 (16/02/2018 - 02/03/2018)}

Primer sprint del proyecto. En la la reunión de planificación de este primer sprint se discute de forma general de lo que va a tratar el proyecto. 

Las tareas realizadas durante este sprint fueran la creación y configuración del repositorio y sobre todo investigar sobre las tecnologías y herramientas que se podían utilizar.

Se investigaron posibles fuentes de datos para implementar más adelante: INE, sepe, aeat.

La duración fue de dos semanas en lugar de una para poder documentarse sobre todos los aspectos relevantes del proyecto.

\imagen{sprints/sprint0}{Burndown del sprint 0}

\href{https://github.com/IvanArjona/TFG-Datos-publicos/milestone/1?closed=1}{Tareas del sprint 0 en Github}

\subsection{Sprint 1 (03/03/2018 - 08/03/2018)}

Un objetivo de este sprint es investigar alternativas de bases de datos no relacionades que se podrían utilizar. Se ha elegido MongoDB.

El otro objetivo es empezar a implementar prototipos con las fuentes de datos que se habían encontrado en el sprint anterior. Se implementaron prototipos del INE, de la agencia tributaria y del sepe.

\imagen{sprints/sprint1}{Burndown del sprint 1}

\href{https://github.com/IvanArjona/TFG-Datos-publicos/milestone/2?closed=1}{Tareas del sprint 1 en Github}

\subsection{Sprint 2 (09/03/2018 - 15/03/2018)}

El primer objetivo de este sprint es crear la estructura de la página web con Flask. Utilizando un modelo vista-controlador. También se utiliza bootstrap para ahorrar trabajo en el diseño.

Se implementó un prototipo de la carga de datos hacia la base de datos y otro para la descarga de datos desde la base de datos para ser mostrados.

Se hizo una implementación de las fuentes de datos a partir de los prototipos del sprint 1 de modo que se pueda cargar todas las fuentes de manera automática.

\imagen{sprints/sprint2}{Burndown del sprint 2}

\href{https://github.com/IvanArjona/TFG-Datos-publicos/milestone/3?closed=1}{Tareas del sprint 2 en Github}

\subsection{Sprint 3 (16/03/2018 - 22/03/2018)}

En este sprint se sopesaron varias plataformas para hacer el despliegue de la web. De ellas se eligió DigitalOcean y Nanobox.

Se realizó el despliegue utilizando estas plataformas. \href{http://tfg-datos-publicos.nanoapp.io/}{Web desplagada}.

Se investigaron los posibles riesgos de seguridad como inyecciones NoSQL.

Se implementó un formulario para la consulta de datos en la página web. En esta primera aproximación se podía hacer una consulta comparando con una columna de una de las fuentes de datos.

\imagen{sprints/sprint3}{Burndown del sprint 3}

\href{https://github.com/IvanArjona/TFG-Datos-publicos/milestone/4?closed=1}{Tareas del sprint 3 en Github}

\subsection{Sprint 4 (23/03/2018 - 13/04/2018)}

Este sprint coincide con semana santa, por lo que dura una semana más de lo habitual y la carga de trabajo también es mayor.

Se corrigieron errores en los tipos de las fuentes de datos al tratar con números como cadenas. 

Se implementó una forma de descargar las consultas a partir del formulario. 

Se mejoró interfaz gráfica y el formulario de consulta.

Se modificaron las fuentes de datos para corregir errores y añadir el código de municipio a todas ellas para más tarde poder unirlas.

\imagen{sprints/sprint4}{Burndown del sprint 4}

\href{https://github.com/IvanArjona/TFG-Datos-publicos/milestone/5?closed=1}{Tareas del sprint 4 en Github}

\subsection{Sprint 5 (14/04/2018 - 25/04/2018)}

Este sprint se dedicó a empezar a documentar la memoria y corregir algunos errores en la página de consulta como el reenvio de formularios en firefox y la descarga de consultas en json y csv.

También se añadió una descripción a la fuente de datos para explicar de qué se trata cada una en la interfaz web.

\imagen{sprints/sprint5}{Burndown del sprint 5}

\href{https://github.com/IvanArjona/TFG-Datos-publicos/milestone/6?closed=1}{Tareas del sprint 5 en Github}

\subsection{Sprint 6 (26/04/2018 - 02/05/2018)}

El objetivo principal de este sprint fue implementar la posibilidad de juntar varias subconsultas mediante un join. Otra característica implementada es la de avisar al usuario si se ha sobrepasado el límite de columnas especificado, para que pueda filtrar más fino.

También se consideró hacer cambios en el modelo de datos, pero debido a la alta dimensionalidad se ha dejado como estaba en el sprint anterior.

\imagen{sprints/sprint6}{Burndown del sprint 6}

\href{https://github.com/IvanArjona/TFG-Datos-publicos/milestone/7?closed=1}{Tareas del sprint 6 en Github}

\subsection{Sprint 7 (03/05/2018 - 09/05/2018)}

En este sprint se implementó un mapa coroplético para mostrar los valores de cualquier atributo en el mapa agrupando los municipios por su provincia.

Se eliminar los campos duplicados de las consultas que surgían al realizar join de varias subconsultas. Como estos campos repetidos siempre son iguales, se ha optado por eliminarlos en lugar de renombrarlos.

\imagen{sprints/sprint7}{Burndown del sprint 7}

\href{https://github.com/IvanArjona/TFG-Datos-publicos/milestone/8?closed=1}{Tareas del sprint 7 en Github}

\section{Estudio de viabilidad}

\subsection{Viabilidad económica}

\subsection{Viabilidad legal}



\apendice{Especificación de Requisitos}

\section{Introducción}

En este apéndice se describirán los objetivos de la aplicación y se detallarán tanto los requisitos funcionales como los no funcionales.

\section{Objetivos generales}

\begin{itemize}
	\item Integrar varias fuentes de datos públicas en una única base de datos. Estos datos son datos de carácter sociológicos, económicos y demográfico a nivel municipal en España.
	\item Permitir añadir nuevos conjunto de datos de forma sencilla.
	\item Desarrollar un algoritmo para la carga de varias fuentes de datos en una base de datos de manera automatizada.
	\item Desarrollo de una aplicación web que permita la consulta de los datos de manera sencilla y visual.
	\item Facilitar la interpretación de los datos utilizando un mapas coropléticos interactivo.
	\item Desplegar la aplicación web en un servidor de forma que sea fácil de actualizar cada vez que se realice un cambio. Además de funcionar en un entorno local.
\end{itemize}

\section{Catalogo de requisitos}

\subsection{Requisitos funcionales}

\begin{itemize}
	\item \textbf{RF-1 Cargar datos}: El administrador de datos debe poder cargar y actualizar los datos desde sus respectivas fuentes de forma automatizada.
	\item \textbf{RF-2 Consulta}: Los usuarios deben poder consultar información de las fuentes de datos.
	\begin{itemize}
		\item \textbf{RF-2.1}: Podrán realizarse varias subconsultas al mismo tiempo.
		\item \textbf{RF-2.2}: Se podrá seleccionar las columnas resultantes a mostrar.
		\item \textbf{RF-2.3}: Se podrá filtrar según los campos de una de las fuentes de datos.
	\end{itemize}
	\item \textbf{RF-3 Visualizar datos}: Los usuarios deben poder visualizar los datos de una consulta en forma de mapa coroplético.
	\begin{itemize}
		\item \textbf{RF-3.1}: Los datos podrán elegirse de cualquier columna numérica.
		\item \textbf{RF-3.2}: Se podrán mostrar los datos en un mapa a nivel municipal o provincial.
		\item \textbf{RF-3.3}: Los datos de una columna se agregarán utilizando varios métodos (media, suma y cuenta).
	\end{itemize}
	\item \textbf{RF-4 Exportar}: Los usuarios deben poder exportar datos en varios formatos.
	\item \textbf{RF-5 Juntar csv}: Los usuarios deben poder juntar varios archivos csv.
		\begin{itemize}
		\item \textbf{RF-5.1}: Juntar los archivos mediante una columna común.
		\item \textbf{RF-5.2}: Utilizando varios tipos de join (\textit{inner}, \textit{outer}, \textit{left}, \textit{right}).
		\item \textbf{RF-5.3}: Seleccionar la ruta donde se exporta el resultado.
	\end{itemize}
\end{itemize}

\subsection{Requisitos no funcionales}

\begin{itemize}
	\item \textbf{RNF-1 Usabilidad}: La aplicación debe ser fácil de usar e intuitiva para el usuario.
	\item \textbf{RNF-2 Rendimiento}: La aplicación debe cargar en un tiempo aceptable.
	\item \textbf{RNF-3 Mantenimibilidad}: La aplicación debe permitir añadir características de forma sencilla.
	\item \textbf{RNF-4 Compatibilidad}: La aplicación debe funciona correctamente en los navegadores modernos más utilizados (Edge, Chrome, Firefox, Opera y Safari).
	\item \textbf{RNF-5 Responsividad}: La aplicación debe funcionar en pantallas de cualquier tamaño y adaptar su interfaz a cada pantalla.
	\item \textbf{RNF-6 Escalabilidad}: La aplicación debe poder aumentar su rendimiento al aumentar recursos hardware.
	\item \textbf{RNF-7 Facilidad de despliegue}: La aplicación debe poder desplegarse en un servidor de forma sencilla.
	\item \textbf{RNF-8 Software libre}: Utilizar software libre siempre que sea posible.
\end{itemize}

\section{Especificación de requisitos}

\subsection{Diagrama de casos de uso}

\subsection{Descripcion de casos de uso}



\apendice{Especificación de diseño}

\section{Introducción}

En este anexo se explican los diseños que se han llevado a cabo para realizar los objetivos anteriores. Se incluye el diseño de datos, diseño procedimental y diseño arquitectónico.

\section{Diseño de datos}

La base de datos utilizada en este proyecto es una base de datos de tipo NoSQL, por lo que no hay un sistema relacional de tablas.

Como en la aplicación se trabaja con datos públicos, no se consideró realizar un sistema de usuarios, por lo que cualquiera puede usarla.

\subsection{Fuentes de datos}

Las fuentes de datos se almacenan cada una de ellas en una colección con sus campos correspondientes.

Todas estas fuentes de datos tiene dos campos comunes: código de municipio y código de comunidad. Estos campos se utilizan para agregar varias fuentes de datos en la consulta y dibujar los mapas coropléticos por provincia o municipio.

% TODO Diagrama entidad relación?

\subsection{Fronteras geográficas}

Para poder dibujar los mapas coropléticos es necesario tener almacenados los límites geográficos para pintar las porciones correspondientes.

Estos límites geográficos están almacenados an archivos \textit{geojson} \cite{misc:geojson}, que son estructuras siguiendo el formato \textit{json} para representar elementos geográficos sencillos.

Los geojson utilizados se han obtenido de \cite{misc:limitesmunicipales} en el caso de los límites geográficos por provincias y \cite{misc:carto} para los límites municipales.

En ambos casos se ha utilizado \foothref{mapshaper}{http://mapshaper.org/} \cite{misc:mapshaper} para minimizar los archivos y que el renderizado sea menos pesado en el navegador.

\section{Diseño procedimental}

\section{Diseño arquitectónico}



\apendice{Documentación técnica de programación}

\section{Introducción}

En este anexo se explica todo lo que tiene que conocer el programador tanto para instalar y ejecutar la aplicación como para poder seguir con el desarrollo.

\section{Estructura de directorios}

\section{Manual del programador}

\section{Instalación y ejecución del proyecto}

\subsection{Instalación}

\subsubsection{MongoDB}

Antes de empezar con la instalación de la aplicación tenemos que instalar la base de datos.
Se ha utilizado una base de datos no relacional MongoDB.

Concretamente se instaló la versión \textit{3.6.4 Community Server} \cite{misc:mongodb} para windows.

También se ha utilizado \textit{MongoDB Compass} \cite{misc:mongodb} para visualizar el contenido de la base de datos. Esta herramienta es opcional.

\subsubsection{Python}

Esta aplicación se ha desarrollado utilizando la version 3.6.4 \cite{misc:python3}, por lo que se recomienda utilizar esta versión o una posterior. En cualquier caso debe de instalarse Python 3 o superior para evitar incompatibilidades.

Todos los paquetes utilizados en la aplicación están listados en en ficheros requirements.txt junto con sus correspondientes versiones.

Habrá que instalar todas las dependencias utilizando pip\footnote{Gestor de paquetes de Python \cite{misc:pip}}:

\begin{lstlisting}
pip install -r requirements.txt
\end{lstlisting}

\subsection{Ejecución}

\subsubsection{Actualización de la base de datos}

Antes de ejecutar la aplicación tendremos que construir la base de datos con el contenido de todas nuestras fuentes por primera vez.

Para ello simplemente hay que ejecutar el fichero \textit{actualiza-fuentes.py}. Puede tardar un rato debido a la gran cantidad de datos que se van a cargar.

Este paso puede repetirse cada vez que se quiera actualizar las fuentes de datos. Puede ser utilizar ejecutarlo una vez al mes para mantener al día las fuentes con datos mensuales.

\begin{lstlisting}
python actualiza-fuentes.py
\end{lstlisting}

\subsubsection{Servidor web}

El servidor Flask ya se ha instalado como un paquete, por lo que no necesita más instalación.
Para ejecutarlo hay un script \textit{run.py} que nos lanza el servidor en el puerto 5000.

Una vez lanzado podremos acceder a la página web desde \href{http://localhost:5000}{localhost:5000}.

\begin{lstlisting}
python run.py
\end{lstlisting}

\section{Despliegue}

Para desplegar la aplicación se ha obtado utilizar como servidor en la nube DigitalOcean y Nanobox como microservicio para facilitarnos la instalación de la máquina en la nube y la configuración del servidor.

\subsection{Instalación}

\subsubsection{Digital Ocean}

Primero tendremos que registrarnos en \href{https://www.digitalocean.com/}{DigitalOcean} y obtener el token de nuestro usuario. Tendremos que poner una tarjeta de crédito de la que nos cobrarán el importe del servidor.

Podría utilizarse otro proveedor de hosting como \textit{Amazon Web services} o \textit{Google Compute}.

\subsubsection{Nanobox}

Después nos registraremos en \href{https://nanobox.io/}{Nanobox} y creamos una nueva aplicación utilizando el token que hemos obtenido antes en DigitalOcean.

Ahora elegiremos el plan que queramos contratar. Para este proyecto será suficiente con el plan más basico de 5\$, 1 CPU y 1GB de ram.

Tras crear la aplicación instalamos el \href{https://docs.nanobox.io/install/}{cliente de nanobox}. La primera vez que lancemos un comando nos pedirá los datos para iniciar sesión en nuestra cuenta.

En el fichero \textit{nanobox.yml} tenemos la configuración con los componentes que se van a instalar y los comandos que se ejecutan al desplegar la aplicación.

\subsection{Despliegue}


Una vez todo instalado y configurado, cada vez que queramos actualizar la aplicación del servidor nos situamos en la carpeta del proyecto y lanzamos el siguiente comando:

\begin{lstlisting}
nanobox deploy
\end{lstlisting}

Con esto, de forma transparente para nosotros, se crea una máquina virtual con nuestro proyecto en la que se instalan la base de datos, el entorno de ejecución y los paquetes y se ejecuta la aplicación en el servidor.

\section{Pruebas del sistema}

\apendice{Documentación de usuario}

\section{Introducción}

En este apéndice se explica cuáles son los requisitos que deberá cumplir el usuario para ejecutar la aplicación, como instalarla y cómo utilizarla.

\section{Requisitos de usuarios}

Como se trata de una aplicación web, el único requisito que necesita el usuario es un navegador web que soporte, \textit{Javascript}, \textit{cookies} y hojas de estilo \textit{CSS}. Como nuestra aplicación también utiliza \textit{jQuery}, nuestros navegadores soportados serán los mismos \cite{misc:jquerybrowsers}:

\begin{itemize}
	\item Google Chrome.
	\item Microsoft Edge.
	\item Mozilla Firefox.
	\item Internet Explorer 9 o superior.
	\item Safari para Mac.
	\item Opera.
	\item Navegador de Android 4.0 o superior.
	\item Safari para iOS 7 o superior.
\end{itemize}

En el caso de la aplicación para juntar ficheros CSV, haría falta tener instalado Python 3 y \textit{Pandas 0.23}.

\section{Instalación}

\subsection{Aplicación web}

Como se trata de una aplicación web, los usuarios no tendrán que realizar ninguna instalación. Sólo serían necesario acceder a la aplicación desde la siguiente url:

\href{https://tfg-datos-publicos.nanoapp.io}{https://tfg-datos-publicos.nanoapp.io}

En caso de que se quisiera instalar la aplicación en local habría que seguir los pasos explicados en la documentación para programador (\ref{instalacionprogramador}). 

\subsection{Aplicación para juntar CSV}

La otra herramienta proporcionada en este trabajo es la aplicación para juntar ficheros CSV. En este caso, tampoco haría falta instalación como tal, pero si tener instalado Python 3 y Pandas 0.23. Una forma rápida de instalarlo es usando el instalador de \textit{Anaconda}\footnote{Anaconda: \href{https://www.anaconda.com/download/}{https://www.anaconda.com/download/}} con \textit{Python 3.6}.

Para ejecutar la aplicación:

\begin{lstlisting}
python joincsv.py
\end{lstlisting}

\section{Manual del usuario}





\bibliographystyle{plain}
\bibliography{bibliografiaAnexos}

\end{document}
