\capitulo{3}{Conceptos teóricos}

A continuación se van a explicar los conceptos teóricos más importantes del proyecto, de modo que se pueda entender con exactitud el proyecto en su totalidad.

\section{Datos públicos}

Se entiende como datos públicos o datos abiertos aquellos que deben estar disponibles de manera libre, para acceder, utilizar, modificar y publicar sin restricciones de copyright \cite{misc:datospublicos}.

Los gobiernos tienen la capacidad de obtener grandes cantidad de información sobre la población a través de varios organismos (como podría ser el \textit{Instituto Nacional de Estadística}). Cuando estos gobiernos liberan los datos para que cualquiera pueda utilizarlos libremente se conocen como `Open Government Data'.

En este proyecto nos centraremos en los datos públicos españoles, aprovechando la `Iniciativa Aporta' que promueve la apertura de información en el sector público en España. Esta iniciativa tiene el objetivo de favorecer el desarrollo de la reutilización de la información del sector público y ayudar a las administraciones para que publiquen sus datos de acuerdo al marco legislativo vigente \cite{misc:iniciativaaporta}.

Aunque nos vamos a centrar en datos proporcionados por organismos estatales, también podrían emplearse datos públicos proporcionados por empresas privadas (Por ejemplo \textit{Google Trends} o datos meteorológicos), aunque esto último se dejará pendiente para trabajos futuros.

\section{Bases de datos no relacionales}

Las bases de datos no relacionales \cite{wiki:nosql}, también llamadas NoSQL (`Not Only SQL') son bases de datos optimizadas para ser utilizadas con modelos de datos sin esquema y potencialmente escalables.

A diferencia de las bases de datos relacionales, aquí generalmente no hay tablas, esquemas, ni relaciones, sino que los datos pueden almacenarse con cualquier esquema sin tener que seguir todos la misma estructura.

En concreto para este proyecto se han optado por bases de datos no relacionales de documentos. Este tipo de bases de datos permiten almacenar documentos en formatos como XML y JSON. Tenemos colecciones en lugar de tablas y dentro de estas colecciones tenemos documentos. Estos documentos pueden tener distintos esquemas entre ellos.

Una de las razones importantes por las que usar bases de datos NoSQL, en especial de documentos, es la especialización del trabajo con agregaciones, la distribución en la nube y la disponibilidad.

\section{Web scraping} \label{webscraping}

Web Scraping \cite{wiki:webscraping} es una técnica para extraer información de sitios web directamente de su código fuente sin utilizar APIs\footnote{Interfaz de programación de aplicaciones} proporcionadas por el propio sitio. La razón de utilizar estas técnicas es que no todos los sitios web nos proporcionan APIs públicas que podamos utilizar.

Lo que hacemos con un \textit{web scraper} es buscar información dentro de un documento web siguiendo ciertos patrones en la estructura de su código fuente y extraer esta información a nuestro entorno local.

Las técnicas de web scraping se centran en transformar datos sin estructura de una página web en datos estructurados para poder ser almacenados y analizados posteriormente, por ejemplo, en una base de datos, hojas de cálculo o en dataframes de Pandas.

\imagencite{conceptos/webscraping}{Esquema del funcionamiento de web scraping}{misc:webscraping}

\section{Mapas coropléticos} \label{mapascoropleticos}

Un mapa coroplético \cite{wiki:mapascoropleticos} es un mapa topológico dividido en regiones en el que cada una de estas regiones se pinta de un color de acuerdo a una medida estadística.

Para ilustrarlo mejor pondremos el ejemplo del siguiente mapa en la figura \ref{fig:conceptos/mapa-paro}. En él se compara el paro de todas las provincias de España pintando con colores más cálidos las provincias con mayor porcentaje de paro y con colores más fríos las provincias con menor porcentaje de paro.

\imagencite{conceptos/mapa-paro}{Mapa coroplético del paro en las provincias españolas}{art:mapaparo}

