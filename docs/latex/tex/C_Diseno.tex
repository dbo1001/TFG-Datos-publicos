\apendice{Especificación de diseño}

\section{Introducción}

En este anexo se explican los diseños que se han llevado a cabo para realizar los objetivos anteriores. Se incluye el diseño de datos, diseño procedimental y diseño arquitectónico.

\section{Diseño de datos}

La base de datos utilizada en este proyecto es una base de datos de tipo NoSQL, por lo que no hay un sistema relacional de tablas.

Como en la aplicación se trabaja con datos públicos, no se consideró realizar un sistema de usuarios, por lo que cualquiera puede usarla.

\subsection{Fuentes de datos}

Las fuentes de datos se almacenan cada una de ellas en una colección con sus campos correspondientes.

Todas estas fuentes de datos tiene dos campos comunes: código de municipio y código de comunidad. Estos campos se utilizan para agregar varias fuentes de datos en la consulta y dibujar los mapas coropléticos por provincia o municipio.

\subsection{Fronteras geográficas}

Para poder dibujar los mapas coropléticos es necesario tener almacenados los límites geográficos para pintar las porciones correspondientes.

Estos límites geográficos están almacenados an archivos \textit{GeoJSON} \cite{misc:geojson}, que son estructuras siguiendo el formato \textit{JSON} para representar elementos geográficos sencillos.

Los GeoJSON utilizados se han obtenido de \cite{misc:limitesmunicipales} en el caso de los límites geográficos por provincias y \cite{misc:carto} para los límites municipales.

En ambos casos se ha utilizado \foothref{mapshaper}{http://mapshaper.org/} \cite{misc:mapshaper} para minimizar los archivos y que el renderizado sea menos pesado en el navegador.

\section{Diseño procedimental}

\section{Diseño arquitectónico}

El diseño de esta aplicación está dividido en dos partes, las fuentes de datos y la aplicación web. Cada una de ellas situada en un paquete.

A continuación se explicará con más detalle cada uno de estos apartados.

\subsection{Fuentes de datos}

Este paquete contiene los ficheros necesarios para guardar los datos desde cada una de las fuentes en la base de datos.

En primer lugar hay una clase abstracta \textit{Fuente} que contiene las operaciones comunes a todas las fuentes de datos y métodos para acceder a el nombre de la colección y la descripción.

De esta clase \textit{Fuente} heredan todas las fuentes de datos, en el diagrama se muestra la clase \textit{Sepe} para no complicarlo demasiado mostrando todas. Esta clase implementa la lógica común a todos los datos obtenidos de esta fuente, en este caso el SEPE. Se implementa el método \textit{carga()} heredado de Fuente, en este método se guarda en un dataframe de Pandas los datos procesados de esta fuente.

De la clase de cada fuente de datos heredan las diferentes consultas que vayamos a almacenar. Por ejemplo para el SEPE tenemos datos de contratos, empleo y paro; cada una en una clase. Esta clase lo que hace es parametrizar a su superclase, por ejemplo, con información de la url base donde se consulta y la lista de años que se usará para acceder a varias urls con la misma estructura dependiendo del año.

Por último, hay que explicar que la función \textit{actualiza\_fuentes} es la encargada de acceder a todas las fuentes, recoge el dataframe de Pandas con la función \textit{carga()} y guarda estos dataframes en la base de datos.

En la figura \ref{fig:diagramas/fuentes} se muestra un diagrama de clases explicado anteriormente.

\imagengrande{diagramas/fuentes}{Diagrama de clases del paquete Fuentes}

\subsection{Aplicación web}

Este paquete está situado en el directorio `web'.

Para el diseño de la aplicación web se ha seguido un patrón Modelo--Vista--Controlador. Al utilizar este patrón conseguimos tener una separación entre la vista, que es la parte con la que interactúa el usuario; la lógica que interacciona con la aplicación y el modelo.

\imagencite{diagramas/mvc}{Diagrama Modelo--Vista--Controlador}{wiki:mvc}

Se ha seguido la estructura seguida en el libro Flask \textit{Web Development de Miguel Grinberg} \cite{book:flask}. En este proyecto los modelos se encuentra en la carpeta \textit{forms}, las vistas en la subcarpeta \textit{template} y los controlador en la raíz de la aplicación web.
