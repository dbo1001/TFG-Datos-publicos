\apendice{Especificación de Requisitos}

\section{Introducción}

En este apéndice se describirán los objetivos de la aplicación y se detallarán tanto los requisitos funcionales como los no funcionales.

\section{Objetivos generales}

\begin{itemize}
	\item Integrar varias fuentes de datos públicas en una única base de datos. Estos datos son datos de carácter sociológicos, económicos y demográfico a nivel municipal en España.
	\item Permitir añadir nuevos conjunto de datos de forma sencilla.
	\item Desarrollar un algoritmo para la carga de varias fuentes de datos en una base de datos de manera automatizada.
	\item Desarrollo de una aplicación web que permita la consulta de los datos de manera sencilla y visual.
	\item Facilitar la interpretación de los datos utilizando un mapas coropléticos interactivo.
	\item Desplegar la aplicación web en un servidor de forma que sea fácil de actualizar cada vez que se realice un cambio. Además de funcionar en un entorno local.
\end{itemize}

\section{Catalogo de requisitos}

\subsection{Requisitos funcionales}

\subsection{Requisitos no funcionales}

\begin{itemize}
	\item \textbf{RNF-1 Usabilidad}: La aplicación debe ser fácil de usar e intuitiva para el usuario.
	\item \textbf{RNF-2 Rendimiento}: La aplicación debe cargar en un tiempo aceptable.
	\item \textbf{RNF-3 Mantenimibilidad}: La aplicación debe permitir añadir características de forma sencilla.
	\item \textbf{RNF-4 Compatibilidad}: La aplicación debe funciona correctamente en los navegadores modernos más utilizados (Edge, Chrome, Firefox, Opera y Safari).
	\item \textbf{RNF-5 Responsividad}: La aplicación debe funcionar en pantallas de cualquier tamaño y adaptar su interfaz a cada pantalla.
	\item \textbf{RNF-6 Escalabilidad}: La aplicación debe poder aumentar su rendimiento al aumentar recursos hardware.
	\item \textbf{RNF-7 Facilidad de despliegue}: La aplicación debe poder desplegarse en un servidor de forma sencilla.
	\item \textbf{RNF-8 Software libre}: Utilizar software libre siempre que sea posible.
\end{itemize}

\section{Especificación de requisitos}

\subsection{Diagrama de casos de uso}

\subsection{Descripcion de casos de uso}


