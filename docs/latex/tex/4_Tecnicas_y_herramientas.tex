\capitulo{4}{Técnicas y herramientas}

\section{Herramientas}

\subsection{MongoDB}

\begin{itemize}
	\tightlist
	\item
	Herramientas consideradas:
	\href{http://basho.com/riak/}{Riak}, 
	\href{http://cassandra.apache.org/}{Cassandra},
	\href{https://www.mongodb.com/}{MongoDB}, 
	\href{http://leveldb.org/}{LevelDB}.
	\item
	Herramienta elegida:
	\href{https://www.mongodb.com/}{MongoDB}.
\end{itemize}

MongoDB es un sistema de bases de datos NoSQL. En esta herramienta los datos se guardan en forma de documentos con un esquema similar a json. Con este sistema se consigue una consulta de datos más rápida.

Tanto Riak como Cassandra son también bases de datos NoSQL, se descartaron porque no ofrecen soportes para equipo con sistema operativo windows. 

LevelDB es una base de datos NoSQL de pares clave-valor, esta herramienta se descartó porque no se cree conveniente utilizar pares clave-valor para un proyecto como este y no hay tantos ejemplos en la documentación como en las otras herramientas.
